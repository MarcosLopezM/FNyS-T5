\documentclass[./../main.tex]{subfiles}
\graphicspath{{img/}}

\begin{document}
    \begin{exercise}
        La interacción \ch{e+ + e- -> e+ + e-} puede suceder de dos formas, dibuja los diagramas de cada una de las posibilidades y checa las conservaciones.

        \begin{solution}
            Antes de dibujar los diagramas de Feynman verificamos las conservaciones. Notemos que esta interacción es un proceso leptónico, por lo que el número bariónico es cero y, además, que tenemos un par de partícula-antipartícula en ambos lados de la interacción, \idest la carga se conserva. De esta manera, lo único que falta verificar es la conservación del número leptónico, tal que,

            \begin{align*}
                \ch{e+ + e- &-> e+ + e-},\\
                \ch{+ \(1_{\e}\) - \(1_{\e}\) &-> + \(1_{\e}\) - \(1_{\e}\)},\\
                \ch{0 &-> 0}.
            \end{align*}

            \setulcolor{pinkwave}\ul{El número leptónico se conserva.}

            Ahora podemos dibujar cada una de las posibilidades:

            \begin{itemize}
                \item Primera posibilidad

                \begin{figure}[htb]
                    \centering
                    \feynmandiagram [horizontal=v1 to v2] {
                        i1 [particle=\(\e^{-}\)] -- [fermion] v1 -- [fermion] i2 [particle=\(\e^{+}\)],
                        v1 -- [photon, edge label=\(\gamma\)] v2,
                        f1 [particle=\(\e^{+}\)] -- [anti fermion] v2 -- [anti fermion] f2 [particle=\(\e^{-}\)],
                    };
                    \caption{Diagrama de Feynman de la primera posibilidad de la interacción.}
                    \label{fig:first-possibility-incomplete}
                \end{figure}                

                \pagebreak
                Aunque la posibilidad es válida, no considera la conservación del momento. El diagrama adecuado debería ser:

                \begin{figure}[htb]
                    \centering
                    \begin{tikzpicture}
                        \begin{feynman}
                            % e+ + e-
                            \vertex (i1) {\(\e^{+}\)};
                            \vertex [below right=of i1] (a);
                            \vertex [below=of a] (b);
                            \vertex [below left=1cm of b] (i2) {\(\e^{-}\)};
                            % fotones
                            \vertex [right=2cm of a] (c);
                            \vertex [right=2cm of b] (d);
                            % e+ + e-
                            \vertex [above right= 1cm of c] (f1) {\(\e^{+}\)};
                            \vertex [below right=1cm of d] (f2) {\(\e^{-}\)};

                            \diagram* [small, horizontal=a to b] {
                                (i1) -- [anti fermion] (a) -- [anti fermion, edge label'=\(\e\)] (b) -- [anti fermion] (i2),
                                (a) -- [photon, edge label=\(\gamma\)] (c),
                                (b) -- [photon, edge label'=\(\gamma\)] (d),
                                (f1) -- [fermion] (c) -- [fermion, edge label=\(\e\)] (d) -- [anti fermion] (f2)
                            };

                        \end{feynman}
                    \end{tikzpicture}
                    \caption{Diagrama de Feynman de la primera posibilidad de la interacción considerando la conservación del momento.}
                    \label{fig:first-possibility-complete}
                \end{figure}  
                
                \item Segunda posibilidad
                
                La segunda y última posibilidad es:

                \begin{figure}[htb]
                    \centering
                    \feynmandiagram [vertical'=a to b] {
                        i1 [particle=\(\e^{+}\)] -- [fermion] a -- [fermion] f1 [particle=\(\e^{+}\)], 
                        a -- [photon, edge label=\(\gamma\)] b,
                        i2 [particle=\(\e^{-}\)] -- [fermion] b -- [fermion] f2 [particle=\(\e^{-}\)],
                    };
                    \caption{Diagrama de Feynman de la segunda posibilidad de la interacción.}
                    \label{fig:second-possibility}
                \end{figure}
            \end{itemize}
        \end{solution}
    \end{exercise}
\end{document}