\documentclass[./../main.tex]{subfiles}
\graphicspath{{img/}}

\begin{document}
    \begin{exercise}
        Determina el radio del ciclotrón necesario para acelerar \(\pi^{+}\) a \SI{10}{\MeV} si se tiene un campo magnético de \SI{2}{\T} (Teslas). Recuerda que la masa debe estar en kilogramos y la energía en Joules para poder usar Teslas dentro de la ecuación.

        \begin{solution}
            Sabemos que la energía máxima de una partícula extraído de un ciclotrón a un radio \(R\) es

            \begin{equation*}
                T_{\text{máx}} = \dfrac{1}{2}\dfrac{(qBR)^{2}}{m}.
            \end{equation*}

            Y queremos conocer el radio del ciclotrón, resolvemos la expresión anterior para \(R\),

            \begin{equation*}
                R = \dfrac{\sqrt{2T_{\text{máx}}m}}{qB}.
                \label{eq:cyclotronRadius}
            \end{equation*}

            Sin embargo, para poder \cref{eq:cyclotronRadius} la massa debe estar \unit{\kg} y la energía en \unit{\J}. Recordemos entonces que los factores de conversión para cada una, respectivamente, son

            \begin{align*}
                \SI{1}{\eV\per\clightsq} &= \SI{1.782661e-36}{\kg},\\
                \SI{1}{\eV} &= \SI{1.602176e-19}{\J}.
            \end{align*}

            Por lo que los valores para la energía \(T_{\text{máx}}\) y la masa del pión \(\pi^{+}\), respectivamente, son:

            \begin{align*}
                T_{\text{máx}} &= \SI{1.602176e-12}{\J},\\
                m_{\pi^{+}} &= \SI{2.495726e-28}{\kg}.
            \end{align*}

            Sustituyendo los valores correspondientes en \cref{eq:cyclotronRadius} tenemos que

            \begin{align*}
                R &= \dfrac{\sqrt{2(\SI{1.602176e-12}{\J})(\SI{2.495726E-28}{\kg})}}{(\SI{1.602176e-19}{\C})(\SI{2}{\T})}\\
                \Aboxedmain{R &= \SI{0.088}{\m}.}
            \end{align*}
        \end{solution}
    \end{exercise}
\end{document}