\documentclass[./../main.tex]{subfiles}
\graphicspath{{img/}}
\begin{document}
	\begin{exercise}
		¿Es posible el decaimiento siguiente?

		\begin{equation*}
			\ch{\Sigma- -> \Lambda^{0} + e- + \(\overline{\nu}\)_{\e}}
		\end{equation*}

		De ser posible dibuja su diagrama de Feynman, ¿qué tipo de interacción es?

		\begin{solution}
			Verificamos las conservaciones, tal que para la energía tenemos

			\begin{equation*}
				\ch{\(\qty{1197.449}{\MeV}\) -> \(\qty{1115.683}{\MeV}\) + \(\qty{0.511}{\MeV}\) + 0}
			\end{equation*}

			La energía del lado izquierdo es mayor, por lo que \setulcolor{pinkwave}\ul{la energía se conserva}. Vemos entonces si la carga se conserva,

			\begin{align*}
				\ch{- \(1\e\) &-> 0 - \(1\e\) + 0},\\
				\ch{- \(1\e\) &-> - \(1\e\)}.
			\end{align*}

			\setulcolor{pinkwave}\ul{La carga se conserva}. Ahora veremos si el número bariónico y leptónico se conserva, recordando que \(\Sigma^{-} = dds\) y \(\Lambda^{0} = uds\) son bariones y \(\e^{-},\ \overline{\nu}_{\e}\) son leptones de la misma familia. Así,

			\begin{itemize}
				\item Número bariónico
				
					\begin{align*}
						\ch{+ 1 &-> + 1},\\
						\Aboxedmain{&\therefore \text{El número bariónico se conserva.}}
					\end{align*}

				\item Número leptónico
				
					\begin{align*}
						\ch{0 &-> + \(1_{\e}\) - \(1_{\e}\)},\\
						\ch{0 &-> 0},\\
						\Aboxedmain{&\therefore \text{El número leptónico se conserva.}}
					\end{align*}
			\end{itemize}

			\pagebreak
			Además por cómo están definidos \(\Sigma^{-}\) y \(\Lambda^{0}\) debemos verificar si se conserva la extrañeza,

			\begin{equation*}
				\ch{- 1 -> - 1}.
			\end{equation*}

			\setulcolor{pinkwave}\ul{La extrañeza se conserva}.

			Dibujamos su diagrama de Feynman,

			\begin{figure}[htb]
				\centering
				\begin{tikzpicture}
					\begin{feynman}
						% d -> d
						\vertex (q1) {\(d\)};
						\vertex [above right=of q1] (a);
						\vertex [right=of a] (q2) {\(d\)};
						% s -> s
						\vertex [below=0.5cm of q1] (q3) {\(s\)};
						\vertex [above right=of q3] (b);
						\vertex [right=of b] (q4) {\(s\)};
						% d -> u
						\vertex [below=0.5cm of q3] (q5) {\(d\)};
						\vertex [above right=of q5] (c);
						\vertex [right=of c] (q6) {\(u\)};
						% Weak interaction
						\vertex [below right=6em of c] (d);
						% electron and electronic neutrino
						\vertex [below right=of d] (f1) {\(\overline{\nu}_{\e}\)};
						\vertex [above right=of d] (f2) {\(\e^{-}\)};

						\diagram* {
							(q1) -- [fermion] (a) -- [fermion] (q2),
							(q3) -- [fermion] (b) -- [fermion] (q4),
							(q5) -- [fermion] (c) -- [fermion] (q6),
							(c) -- [scalar, edge label'=\(W^{-}\)] (d),
							(f1) -- [fermion] (d) -- [fermion] (f2),
						};

						\draw [decoration={brace}, decorate, blue] (q5.south west) -- (q1.north west)
node [pos=0.5, left=0.5cm of q3] {\(\Sigma^{-}\)};
						\draw [decoration={brace}, decorate] (q2.north east) -- (q6.south east)
						node [pos=0.5, right=0.5cm of q4] {\(\Lambda^{0}\)};
					\end{feynman}
				\end{tikzpicture}
				\caption{Diagrama de Feynman para el decaimiento de \(\Sigma^{-}\).}
				\label{fig:sigma-minus-decay}
			\end{figure}
		\end{solution}
	\end{exercise}
\end{document}
