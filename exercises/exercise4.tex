\documentclass[./../main.tex]{subfiles}
\graphicspath{{img/}}
\begin{document}
	\begin{exercise}
		A partir del modelo de capas prediga el momento angular nuclear y la paridad de los siguientes núcleos:

		\begin{itemize}
			\item \ch{^{3}He}
			
			\begin{solution}

				Tenemos que el número de protones y neutrones es:

				\begin{align*}
					Z &\colon 2 \implies \text{Apareado},\\
					N &\colon 1.
				\end{align*}

				Así el llenado de capas de neutrones se ve como
				
				\begin{figure}[htb]
                    \centering
                    \documentclass[tikz, preview]{standalone}
\usepackage{xcolor}
\definecolor{pinkwave}{RGB}{255, 0, 128}%
\definecolor{complement}{RGB}{0, 255, 127}%
\definecolor{base3}{RGB}{253, 246, 227}%
\pagecolor{base3}
\begin{document}
    \begin{tikzpicture}[baseline=(current bounding box.center)]
        % Energy levels
        \node[anchor=south, label={$2j + 1$}, minimum size=8pt] at (3.8, -1.3) {};
        \node[anchor=south, label={$\sum(2j + 1$)}, minimum size=8pt] at (5.5, -1.3) {};

        \foreach \y/\label/\neutrons/\numberlabel in {0/1s$_{1/2}$/2/2} {
            \draw (0,\y) -- (4,\y);
            \node[anchor=east] at (0,\y) {\label};
            \node[anchor=west] at (4,\y) {\neutrons};
            \node[anchor=west] at (4.5, \y) {\numberlabel};
        }

        % Magic number
        \foreach \y/\magic in {0.5/2} {
            \node at (2, \y) [circle,draw, label={[label distance=-0.2cm]center:\magic}, minimum size=0.6cm] {};
        }
        
        % Neutrons filled states
        \foreach \y/\capacity in {0} {
            \ifnum\y=0
                \foreach \x in {1} {
                    \filldraw[pinkwave] ({(\x - 1) * (3.5 - 0.5) / (\capacity - 1) + 0.5}, \y) circle (0.1);
                    % \draw[complement] ({(\x - 1) * (3.5 - 0.5) / (\capacity - 1) + 0.5}, \y) circle (0.1);
                }
                \draw (3.5, 0) circle (0.1);
            \else
                \foreach \x in {1,...,\capacity} {
                    \filldraw[pinkwave] ({(\x - 1) * (3.5 - 0.5) / (\capacity - 1) + 0.5}, \y) circle (0.1);
                    % \draw[complement] ({(\x - 1) * (3.5 - 0.5) / (\capacity - 1) + 0.5}, \y) circle (0.1);
                }
            \fi
        }
    \end{tikzpicture}
\end{document}

                    \caption{Llenado de niveles para el estado base de \ch{^{3}He}.}
                    \label{fig:FirstExcitedState3He}
                \end{figure}

				Por lo que,

				\begin{align*}
					N &\colon (1s_{1/2})^{1} \implies \ell = 0,\\
					\pi &= (-1)^{0} = +1,\\
					J &= \tfrac{1}{2},\\
					\Aboxedmain{J^{\pi} &= \tfrac{1}{2}^{+}.}
				\end{align*}

				Comparando con el valor experimental de \(J\) que se encuentra en \href{http://easyspin.org/documentation/isotopetable.html}{EasySpin} notamos que se obtiene el mismo resultado, \(J = \tfrac{1}{2}\).
			\end{solution}
			
			\pagebreak
                \item \ch{^{15}O}
			
			\begin{solution}
				El número de protones y neutrones es

				\begin{align*}
					Z &\colon 8 \implies \text{Apareado},\\
					N &\colon 7.
				\end{align*}

				El llenado de capas se hará con los neutrones, tal que

				\begin{figure}[htb]
                    \centering
                    \documentclass[tikz]{standalone}
\usepackage{xcolor}
\definecolor{pinkwave}{RGB}{255, 0, 128}%
\definecolor{pinkwave}{RGB}{0, 255, 127}%
\definecolor{base3}{RGB}{253, 246, 227}%
\pagecolor{base3}
\begin{document}
    \begin{tikzpicture}[baseline=(current bounding box.center)]
        % Energy levels
        \node[anchor=south, label={$2j + 1$}, minimum size=8pt] at (3.8, -1.3) {};
        \node[anchor=south, label={$\sum(2j + 1$)}, minimum size=8pt] at (5.5, -1.3) {};

        \foreach \y/\label/\neutrons/\numberlabel in {0/1s$_{1/2}$/2/2, 1/1p$_{3/2}$/4/6, 2/1p$_{1/2}$/2/8} {
            \draw (0,\y) -- (4,\y);
            \node[anchor=east] at (0,\y) {\label};
            \node[anchor=west] at (4,\y) {\neutrons};
            \node[anchor=west] at (4.5, \y) {\numberlabel};
        }

        % Magic number
        \foreach \y/\magic in {0.5/2, 2.5/8} {
            \node at (2, \y) [circle,draw, label={[label distance=-0.2cm]center:\magic}, minimum size=0.6cm] {};
        }
        
        % Neutrons filled states
        \foreach \y/\capacity in {0/2, 1/4} {
            \ifnum\y=0
                \foreach \x in {1,2} {
                    \filldraw[pinkwave] ({(\x - 1) * (3.5 - 0.5) / (\capacity - 1) + 0.5}, \y) circle (0.1);
                }
            \else
                \foreach \x in {1,...,\capacity} {
                    \filldraw[pinkwave] ({(\x - 1) * (3.5 - 0.5) / (\capacity - 1) + 0.5}, \y) circle (0.1);
                }
            \fi
        }

        \filldraw[pinkwave] (0.5, 2) circle (0.1);

        % Neutrons empty states
        \foreach \x in {2} {
            \draw ({(\x - 1) * (3.5 - 0.5) / (2 - 1) + 0.5}, 2) circle (0.1);
        }
    \end{tikzpicture}
\end{document}

                    \caption{Llenado de niveles para el estado base de \ch{^{15}O}.}
                    \label{fig:FirstExcitedState15O}
                \end{figure}

				Así,

				\begin{align*}
					N &\colon (1p_{1/2})^{1} \implies \ell = 1,\\
					\pi &= (-1)^{1} = -1,\\
					J &= \tfrac{1}{2},\\
					\Aboxedmain{J^{\pi} &= \tfrac{1}{2}^{-}.}
				\end{align*}

				No podemos comparar el valor de \(J\) predicho con el de la página de \href{http://easyspin.org/documentation/isotopetable.html}{EasySpin} ya que no se encuentra registrado, tal vez debido a que no se ha medido experimentalmente.
			\end{solution} 
			
			\pagebreak
                \item \ch{^{41}Ca}
			
			\begin{solution}
				El número de protones y neutrones es

				\begin{align*}
					Z &\colon 20 \implies \text{Apareado},\\
					N &\colon 21.
				\end{align*}

				El llenado de capas se hará con los neutrones,

				\begin{figure}[htb]
                    \centering
                    \documentclass[tikz]{standalone}
\usepackage{xcolor}
\definecolor{pinkwave}{RGB}{255, 0, 128}%
\definecolor{pinkwave}{RGB}{0, 255, 127}%
\definecolor{base3}{RGB}{253, 246, 227}%
\pagecolor{base3}
\begin{document}
    \begin{tikzpicture}[baseline=(current bounding box.center)]
        % Energy levels
        \node[anchor=south, label={$2j + 1$}, minimum size=8pt] at (3.8, -1.3) {};
        \node[anchor=south, label={$\sum(2j + 1$)}, minimum size=8pt] at (5.5, -1.3) {};

        \foreach \y/\label/\neutrons/\numberlabel in {0/1s$_{1/2}$/2/2, 1/1p$_{3/2}$/4/6, 2/1p$_{1/2}$/2/8, 3/1d$_{5/2}$/6/14, 4/2s$_{1/2}$/2/16, 5/1d$_{3/2}$/4/20, 6/1f$_{7/2}$/8/28} {
            \draw (0,\y) -- (4,\y);
            \node[anchor=east] at (0,\y) {\label};
            \node[anchor=west] at (4,\y) {\neutrons};
            \node[anchor=west] at (4.5, \y) {\numberlabel};
        }

        % Magic number
        \foreach \y/\magic in {0.5/2, 2.5/8, 5.5/20, 6.5/28} {
            \node at (2, \y) [circle,draw, label={[label distance=-0.2cm]center:\magic}, minimum size=0.6cm] {};
        }
        
        % Neutrons filled states
        \foreach \y/\capacity in {0/2, 1/4, 2/2, 3/6, 4/2, 5/4} {
            \ifnum\y=0
                \foreach \x in {1,2} {
                    \filldraw[pinkwave] ({(\x - 1) * (3.5 - 0.5) / (\capacity - 1) + 0.5}, \y) circle (0.1);
                }
            \else
                \foreach \x in {1,...,\capacity} {
                    \filldraw[pinkwave] ({(\x - 1) * (3.5 - 0.5) / (\capacity - 1) + 0.5}, \y) circle (0.1);
                }
            \fi
        }

        \foreach \x in {1} {
            \filldraw[pinkwave] ({(\x - 1) * (3.5 - 0.5) / (8 - 1) + 0.5}, 6) circle (0.1);
        }

        % Neutrons empty states
        \foreach \x in {2,...,8} {
            \draw ({(\x - 1) * (3.5 - 0.5) / (8 - 1) + 0.5}, 6) circle (0.1);
        }
    \end{tikzpicture}
\end{document}

                    \caption{Llenado de niveles para el estado base de \ch{^{41}Ca}.}
                    \label{fig:FirstExcitedState41Ca}
                \end{figure}

				Así,

				\begin{align*}
					N &\colon (1f_{7/2})^{1} \implies \ell = 3,\\
					\pi &= (-1)^{3} = -1,\\
					J &= \tfrac{7}{2},\\
					\Aboxedmain{J^{\pi} &= \tfrac{7}{2}^{-}.}
				\end{align*}

				Comparando el valor de \(J\) con el valor experimental, se obtiene el mismo resultado, \(J = \tfrac{7}{2}\).
			\end{solution}
			
			\pagebreak
                \item \ch{^{56}Fe}
			
			\begin{solution}
				El número de protones y neutrones es

				\begin{align*}
					Z &\colon 26 \implies \text{Apareado},\\
					N &\colon 30 \implies \text{Apareado}.
				\end{align*}

				Y sabemos que para núcleos par-par el espín nuclear total es igual a 0 con paridad positiva, \idest

				\begin{empheq}[box = \color{pinkwave}\fbox]{equation*}
					J^{\pi} = 0^{+}.
				\end{empheq}

				Comparando con el valor de \(J\) registrado en \href{http://easyspin.org/documentation/isotopetable.html}{EasySpin} verificamos que se obtiene lo mismo.
			\end{solution}
		\end{itemize}

		Compare con los valores de \(J\) observados experimentalmente: \url{http://easyspin.org/documentation/isotopetable.html}
	\end{exercise}
\end{document}
