\documentclass[./../main.tex]{subfiles}
\graphicspath{{img/}}

\begin{document}
    \begin{exercise}
        La interacción \ch{e- + e+ -> \pi+ + \pi- + \pi^{0}} puede ser mediada por un boson vectorial \(\omega^{0}\) que tiene un modo de decaimiento dominante \ch{\omega^{0} -> \pi+ + \pi- + \pi^{0}} ¿cómo será el diagrama de Feynman del proceso completo?

        \begin{solution}
            Como primer paso verificamos que todo se conserve, excepto la energía ya que estamos tratando con una interacción. Notemos antes que en la interacción no hay ningún barión, es 0 en ambos lados, \idest el \setulcolor{pinkwave}\ul{número bariónico se conserva}.

            Ahora verificamos si se la carga se conserva:

            \begin{align*}
                \ch{e- + e+ &-> \pi+ + \pi- + \pi^{0}},\\
                \ch{- \(1\e\) + \(1\e\) &-> \(1\e\) - \(1\e\) + \(0\)},\\
                \ch{0 &-> 0}.
            \end{align*}

            \setulcolor{pinkwave}\ul{La carga se conserva}.

            Para el número leptónico, del lado derecho es 0, ya que todas las partículas son mesones, mientras que del lado izquierdo tenemos

            \begin{align*}
                \ch{\(1_{\e}\) - \(1_{\e}\) &-> 0},\\
                \ch{0 &-> 0}.
            \end{align*}

            \setulcolor{pinkwave}\ul{El número leptónico se conserva} y la interacción es posible.

            Antes de dibujar el diagrama de Feynman recordemos la composición en quarks de cada unos de los piones:

            \begin{align*}
                \pi^{+} &: u\overline{d},\\
                \pi^{-} &: d\overline{u},\\
                \pi^{0} &: u\overline{u},\ d\overline{d}.
            \end{align*}

            \pagebreak
            Entonces,

            \begin{figure}[htb]
                \centering
                \begin{tikzpicture}
                    \begin{feynman}
                        % Left side interaction
                        \vertex (i1) {\(e^{-}\)};
                        \vertex (a) [below right=of i1];
                        \vertex [below left=1cm of a] (i2) {\(e^{+}\)};
                        % Photon
                        \vertex [right=of a] (b);
                        % Loop
                        \vertex [right=of b] (c);
                        \vertex[right=0.75cm of b] (w0);

                        % omega0 decay
                        %%% pi+
                        \vertex [blue, above right=3cm and 2.5cm of c] (adpp) {\(\overline{d}\)}; %anti-quark d of pi+
                        \vertex [blue, below=of adpp] (upp) {\(u\)}; %quark u of pi+

                        %%% p0
                        \vertex [above right=0.5cm and 2.5cm of c] (up) {\(u\)}; %quark u of pi0
                        \vertex [below=of up] (aup) {\(\overline{u}\)}; %anti-quark u of pi0

                        %%% pi-
                        \vertex [pinkwave, below right=1.5cm and 2.5cm of c] (aupm) {\(\overline{u}\)}; %quark d of pi-
                        \vertex [pinkwave, below=of aupm] (dpm) {\(d\)}; %anti-quark u of pi-

                        %%% First pair
                        \vertex [above=0.2cm of c] (d);
                        \vertex [above right=1cm and 1.75cm of d] (e);

                        %%% Second pair
                        \vertex [below=0.2cm of c] (f);
                        \vertex [below right=1cm and 1.75cm of f] (g);

                        %%% Anti-quark d
                        \vertex [above left=0.6cm and 0.5cm of c] (h);
                        \vertex [above=of e] (v1);

                        %%% Quark d
                        \vertex [below left=0.6cm and .5cm of c] (i);
                        \vertex [below=of g] (v2);
                        
                        \diagram* [horizontal=b to c] {
                            (i1) -- [fermion] (a) -- [fermion] (i2),
                            (a) -- [photon, edge label=\(\gamma\)] (b),
                            (b) -- [fermion, half left] (c)
                            -- [fermion, half left] (b),
                            (d) -- [photon, edge label=\(\gamma\)] (e),
                            (f) -- [photon, edge label'=\(\gamma\)] (g),
                            (e) -- [fermion] (upp),
                            (e) -- [fermion] (up),
                            (g) -- [fermion] (aup),
                            (g) -- [fermion] (aupm),
                            (h) -- [photon, edge label=\(\gamma\)] (v1),
                            (i) -- [photon, edge label'=\(\gamma\)] (v2),
                            (v1) -- [fermion] (adpp),
                            (v2) -- [fermion] (dpm),                       
                        };

                        \node at (w0) {\(\omega^{0}\)};
                        \draw [decoration={brace}, decorate, blue] (adpp.north east) -- (upp.south east)node [pos=0.5, right] {\(\pi^{+}\)};
                        \draw [decoration={brace}, decorate] (up.north east) -- (aup.south east) node [pos=0.5, right] {\(\pi^{0}\)};
                        \draw [decoration={brace}, decorate, pinkwave] (aupm.north east) -- (dpm.south east) node [pos=0.5, right] {\(\pi^{-}\)};

                    \end{feynman}
                \end{tikzpicture}   
                \caption{Diagrama de Feynman de la interacción \ch{e- + e+ -> \pi+ + \pi- + \pi^{0}}.}
                \label{fig:e-e+--w0}
            \end{figure}
        \end{solution}
    \end{exercise}
\end{document}