\documentclass[./../main.tex]{subfiles}
\graphicspath{{img/}}

\begin{document}
    \begin{exercise}
        Un fotón de \qty{35}{\MeV} pasa por una dispersión de Compton y sale con un ángulo de \(\tfrac{\pi}{3}\). ¿Cuál es la energía del fotón al salir? ¿Cuál es la energía cinética del electrón dispersado?

        \begin{solution}
            Para conocer la energía del fotón al salir usamos la relación dada por:

            \begin{equation*}
                h\nu^{\prime} = \dfrac{h\nu}{1 + \gamma(1 - \cos\theta)},
            \end{equation*}

            con \(\gamma = \tfrac{h\nu}{m_{\e}c^{2}}\). Así,

            \begin{align}
                h\nu &= \dfrac{\qty{35}{\MeV}}{1 + (\tfrac{35}{0.511})(1 - \cos(\tfrac{\pi}{3}))}\nonumber,\\
                &= \qty{0.993009}{\MeV},\nonumber\\
                \Aboxedmain{h\nu &\simeq \qty{1}{\MeV}.}\label{eq:PhotonEnergyAfterCollision}
            \end{align}

            Por otro lado, para saber la energía cinética del electrón disparado usamos que

            \begin{equation*}
                T_{\e} = h\nu - h\nu^{\prime}.
            \end{equation*}

            Sustituyendo \cref{eq:PhotonEnergyAfterCollision} en la expresión para \(T_{\e}\),

            \begin{align*}
                T_{\e} &= \qty{35}{\MeV} - \qty{1}{\MeV},\\
                \Aboxedmain{T_{\e} &= \qty{34}{\MeV}.}
            \end{align*}
        \end{solution}
    \end{exercise}
\end{document}