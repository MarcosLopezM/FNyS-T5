\documentclass[./../main.tex]{subfiles}
\graphicspath{{img/}}

\begin{document}
    \begin{exercise}
        Mencionan dos tipos de detectores de ionización y explica la base de su funcionamiento.

        \begin{solution}
            
            Los detectores de ionización son dispositivos diseñados para medir la ionización producida cuando una partícula cargada atraviesa un medio material.

            La partícula cargada pasa produciendo ionizaciones en el medio, y el campo eléctrico evita que los pares electrón-ion se recombinen acelerándolos a sus respectivos electrodos. Algunos de los detectores de ionización son: cámaras de ionización y contadores proporcionales.

            \begin{itemize}
                \item Cámara de ionización
                
                Las cámaras de ionización se pueden considerar como un capacitor de placas paralelas, donde la región entre las placas está llena de gas, habitualmente aire. El campo eléctrico en esta región evita que los iones se recombinen con los electrones y los acelera hacia la placa positiva, mientras que los iones se aceleran hacia la otra.

                Las señales producidas son muy pequeñas, por lo que deben ser amplificadas por un factor de alrededor de \num{e4} antes de poder ser analizada. La amplitud de esta señal es proporcional al número de iones formados y es independiente al voltaje entre las placas. Aunque el voltaje aplicado determina la velocidad con la que los iones y electrones son acelerados hacia los electrodos.
                
                \item Contadores proporcionales
                
                Los contadores proporcionales son similares a las cámaras de ionización, pero operan en la región proporcional con campo eléctrico intensos (\(\sim \qty{e4}{\V\per\cm}\)) y amplificaciones alrededor de \num{e5} electrones por ionización. Estos detectores suelen ser arreglos cilíndricos con un alambre central como ánodo y una carcasa cilíndrica como cátodo. La geometría de estos detectores, así como el voltaje aplicado son tales que en casi toda la cámara la magnitud del campo eléctrico es débil y la cámara actúa como una cámara de ionización. Sin embargo, cerca del eje axial, donde se ubica el ánodo, suceden la mayor parte de las multiplicaciones, la avalancha que se produce aquí suele llamarse \emph{avalancha de Townsed}.

                Un diseño clave de estos detectores es que en cada punto de ionización debida a la radiación ionizante produzca una única avalancha. Esto se hace para asegurar la proporcionalidad entre el número de eventos originales y el total de la corriente.
            \end{itemize}
        \end{solution}
    \end{exercise}
\end{document}