\documentclass[./../main.tex]{subfiles}
\graphicspath{{img/}}
\begin{document}
	\begin{exercise}
		Este ejercicio se desdobla en dos, si no deseas hacer la parte de programación solo haz la primera parte, si quieres moverle un poco a la simulación pasa al segundo caso, pero si quieres verte intrépidx, haz los dos para comparar lo que sale:

		\begin{enumerate}
			\item Considera un electrón \SI{20}{\GeV} entrando a la atmósfera, calcula la máxima profundidad que alcanza la cascada electromagnética generada.
			\item Usa la simulación que se encuentra en la página \url{https://marcovladimir.codeberg.page/4tarea.html}, no debe instalar nada, puedes correrla desde \url{https://try.ruby-lang.org/playground/}, solo pon los valores correctos. ¿Qué tipo de distribución siguen las variables aleatorias?
		\end{enumerate}
	\end{exercise}
\end{document}
